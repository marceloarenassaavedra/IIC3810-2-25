\documentclass[10pt]{article}
\usepackage{logo}
\usepackage{fullpage}
\usepackage[utf8]{inputenc}
\usepackage[spanish]{babel}
\usepackage{epsfig}
\usepackage{amsmath}
\usepackage{amssymb}
\usepackage{multicol}
\usepackage{color}

\newcommand{\cyan}[1]{\textCyan #1\textBlack}
\newcommand{\red}[1]{\textRed #1\textBlack}
\newcommand{\green}[1]{\textGreen #1\textBlack}
\newcommand{\blue}[1]{\textBlue #1\textBlack}
\newcommand{\black}[1]{\textBlack #1\textBlack}
\newcommand{\magenta}[1]{\textMagenta #1\textBlack}
\newcommand{\brown}[1]{\textBrown #1\textBlack}
\newcommand{\vs}[1]{\vspace{#1mm}}
\newcommand{\ignore}{}
\newcommand{\ri}[1]{\text{\red{#1}}}
\newcommand{\hs}{\hat\sigma}
\newcommand{\modelos}{{\it modelos}}
\newcommand{\B}{{\tt B}}
\newcommand{\nspace}{\text{NSPACE}}
\newcommand{\logspace}{\text{LOGSPACE}}
\newcommand{\nlogspace}{\text{NLOGSPACE}}
\newcommand{\npspace}{\text{NPSPACE}}
\newcommand{\pspace}{\text{PSPACE}}
\newcommand{\ph}{\text{PH}}
\newcommand{\expspace}{\text{EXPSPACE}}
\newcommand{\nexpspace}{\text{NEXPSPACE}}
\newcommand{\dspace}{\text{DSPACE}}
\newcommand{\espacio}{{\it espacio}}
\newcommand{\tiempo}{{\it tiempo}}
\newcommand{\ptime}{\text{PTIME}}
\newcommand{\dtime}{\text{DTIME}}
\newcommand{\exptime}{\text{EXPTIME}}
\newcommand{\nexptime}{\text{NEXPTIME}}
\newcommand{\CC}{{\cal C}}
\newcommand{\sat}{\text{SAT}}
\newcommand{\tcnfu}{\text{3-CNF-SAT-UNSAT}}
\newcommand{\usat}{\text{unique-SAT}}
\newcommand{\np}{\text{NP}}
\newcommand{\ntime}{\text{NTIME}}
\newcommand{\cnf}{\text{CNF-SAT}}
\newcommand{\tcnf}{\text{3-CNF-SAT}}
\newcommand{\dcnf}{\text{2-CNF-SAT}}
\newcommand{\horn}{\text{HORN-SAT}}
\newcommand{\nhorn}{\text{NEG-HORN-SAT}}
\newcommand{\co}{\text{co-}}
\newcommand{\rp}{\leq^\text{\it p}_\text{\it m}}
\newcommand{\tur}{\leq^\text{\it p}_\text{\it T}}
\newcommand{\reach}{\text{CAMINO}}
\newcommand{\pe}{\text{PROG-ENT}}
\newcommand{\pl}{\text{PROG-LIN}}
\newcommand{\cor}{\text{CONT-REG}}
\newcommand{\er}{\text{EQUIV-REG}}
\newcommand{\qbf}{\text{QBF}}
\newcommand{\shp}{\text{Succinct-HP}}
\newcommand{\hp}{\text{HP}}
\newcommand{\cdp}{\text{DP}}
\newcommand{\clique}{\text{CLIQUE}}
\newcommand{\eclique}{\text{exact-CLIQUE}}
\newcommand{\costo}{\text{costo}}
\newcommand{\tsp}{\text{TSP}}
\newcommand{\tspu}{\text{unique-TSP}}
\newcommand{\no}{\text{NO}}
\newcommand{\yes}{\text{YES}}
\newcommand{\br}{\text{CERTAIN-ANSWERS}}
\newcommand{\dpc}{\text{DP}}

\newcommand{\CROM}{\text{CROM}}
\newcommand{\EVAL}{\text{EVAL}}
\newcommand{\EQUIV}{\text{EQUIV}}

\newcommand{\re}{\text{RE}}
\newcommand{\dec}{\text{R}}
\newcommand{\resp}[1]{{\text{\bf [Responsable: #1]}}}

\newcommand{\GRAPHISO}{\text{GRAPH-ISO}}
\newcommand{\cqbf}{\text{CNF-QBF}}
\newcommand{\am}{\text{AM}}
\newtheorem{lema}{Lema}
\newcommand{\pr}{\text{\rm {\bf Pr}}}
\newcommand{\scnf}{\#\text{CNF-SAT}}
\newcommand{\ucnf}{\text{U-CNF-SAT}}

\begin{document}


\begin{tabular}{ccl}
\begin{tabular}{c}
\psfig{file=puclogo.eps}
\end{tabular}
&\ \ \ & 
\begin{tabular}{l}
PONTIFICIA UNIVERSIDAD CATOLICA DE CHILE\\
ESCUELA DE INGENIERIA\\
DEPARTAMENTO DE CIENCIA DE LA COMPUTACION
\end{tabular}
\end{tabular}

\vspace{1cm}

\begin{center}
\bf Tópicos Avanzados en Teoría de la Computación - IIC3810\\
\bf Tarea 3\\
\bf Fecha de entrega: miércoles 29 de octubre
\end{center}

\vspace{1cm}

\begin{enumerate}
\item En esta pregunta usted debe construir un protocolo interactivo
  con aleatoriedad pública para $\overline{\GRAPHISO}$ a partir del
  protocolo interactivo dado en clases para $\cqbf$.

\item Utilizando las ideas vistas en clases, demuestre que:
  \begin{eqnarray*}
    \bigcup_{k \in \mathbb{N}} \am[n^k] & \subseteq & \pspace.
  \end{eqnarray*}


\item En esta pregunta usted debe completar la demostración del
  teorema de Valiant-Vazirani a partir de siguiente lema demostrado
  en clases:
\begin{lema}
Existe un algoritmo aleatorizado de tiempo polinomial que, dada una
fórmula proposicional $\varphi$ en CNF con $n$ variables, genera una secuencia de
fórmulas $\varphi_1$, $\ldots$, $\varphi_n$, $\varphi_{n+1}$, $\varphi_{n+2}$ en CNF tales que:
\begin{itemize}
\item Si $\varphi$ es consistente, entonces
\begin{eqnarray*}
\pr\bigg(\bigvee_{i=1}^{n+2} \scnf(\varphi_i) = 1\bigg) & \geq & \frac{1}{8}
\end{eqnarray*}

\item Si $\varphi$ es inconsistente, entonces cada fórmula $\varphi_i$ $(i \in \{1, \ldots, n+2\})$ es inconsistente.
\end{itemize}
\end{lema}
En particular, a partir de este lema debe construir una MT probabilística $M$ con oráculo tal que $t_M(n)$ es $O(n^k)$ y para cada
  \begin{align*}
    H \ \subseteq \ \{ \psi \mid \psi \text{ es una fórmula en CNF tal que } \scnf(\psi) \geq 2\}
  \end{align*}
  y cada fórmula $\varphi$ en CNF:
\vs{1}
  \begin{itemize}
\item Si $\varphi \in \cnf$, entonces $\pr(M^{\ucnf_H} \text{ acepte } \varphi) \geq \frac{3}{4}$
\vs{1}
\item Si $\varphi \not\in \cnf$, entonces $\pr(M^{\ucnf_H} \text{ acepte } \varphi) = 0$
\end{itemize}
\end{enumerate}

\end{document}
